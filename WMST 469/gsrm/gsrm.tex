\documentclass[man,12pt]{apa6}

\usepackage[american]{babel}
\usepackage{apacite}


\title{Representations of Gender, Sexual, and Romantic Minorities in Popular Webcomics}
\shorttitle{GSRMs in Webcomics}
\author{Tenzin Beck}
\affiliation{The University of New Mexico}
\leftheader{GSRMs in TV and Webcomics}

\begin{document}
\maketitle

\section{Introduction}
As Larry Gross  argues in ``Out of the Mainstream: Sexual Minorities and the Mass Media'' \citeyear{gross}, mainstream mass media is not only increasingly homogenized, but additionally creates an ideological framework through which we understand the world. Thus, representations of marginalized groups in mainstream media have a potentially huge influence on public perception of such groups. Gross argued that gays and lesbians in particular are subject to both symbolic annihilation and vilification. Such portrayals, he argues, serve to maintain the dominant patriarchal and heteronormative power structure. Additionally, Gross calls for the portrayal of ``just plain gay people'' in media, rather than having gays and lesbians serve simply as exoticized tokens or foils for straight characters.

However, Kelly Kessler \citeyear{queermediocrity} claims that as of 2010, mainstream television has achieved this goal. Kessler 
cites broadcast television shows across multiple genres in which the portrayal of queer individuals and couples are equally fraught with hardship, conflict, and individual failings as heterosexuals within the same shows. To Kessler, this represents a fairness of treatment that normalizes rather than others queerness. 

Whether Kessler's assertion is accurate is a matter of debate, since simply portraying gays and lesbians in a normative context may not be substantially challenging other problematic constructs such as the archetype of gay men as white, affluent, and effeminate. 
Additionally, Kessler does not address the representation of other Gender, Sexual, and Romantic Minorities (GSRMs). This acronym is an umbrella term that is rapidly gaining acceptance in informal online queer spaces (for example, certain reddit.com boards) as an alternative to the increasingly unweildly acronyms describing specific non-normative identities. Additionally, ``Gender, Sexual, and Romantic Minorities'' directly addresses the core unifying aspect of minority status as a result of existing outside of the the scope of traditional hetero-patriarchy. GSRMs include but are not limited to: women, gays and lesbians, trans*\footnote{The construction ``trans*'' is here used since there is ongoing community debate on the relative merits of the words ``transgender'', ``transgendered'', ``transsexual'', and other terms. ``Trans*'' (with the asterisk indicating that any desired suffix may follow) is widely regarded as the most inclusive term.} and gender-nonconforming individuals, bisexuals and pansexuals, people who enjoy non-normative sexual behaviors (kink and/or asexuality), and people in non-normative relationship structures such as open relationships or polyamory.
 
New avenues of media communication are emerging as a result of the omnipresence of the internet. The webcomic (or ``online comic'') is an emerging art form that has enormous potential to create subversive or iconoclastic messages. Webcomic creation is a bottom-up rather than a top-down form of media, characterized by low barriers to entry and enormous degrees of freedom for the creator. Success and reach depends far more on popular or niche appeal than on any corporate backing or managerial control. Webcomic formats range from joke-a-day strips similar to syndicated newspaper comics, to complex storytelling comparable in form to graphic novels, as well as some highly experimental and avant-garde publications that utilize the internet's unique capabilities. Yet another distinguishing feature is that webcomics offer their full archives free of charge. 

Webcomics as a medium are practially unexamined in academic literature. This represents a substantial gap in our understanding of the current media climate, as new forms of creative expression and popular consumption slowly increase their foothold in the public consciousness. 

Therefore, I undertook an investigation the portrayals of GSRMs in some of the most popular comics online.  According to web information company Alexa, in descending order of readership the five most popular webcomics are xkcd (xkcd.com), Penny Arcade (http://penny-arcade.com/comic), Cyanide and Happiness (explosm.net/comics/), Questionable Content (questionablecontent.net), and Saturday Morning Breakfast Cereal (smbc-comics.com) \cite{alexia}. The sample for this study be all strips published by these comics between March 1, 2012 and March 1, 2013.

\section{Review of Literature}

There is little if any scholarly work addressing webcomics as a medium. This represents a significant and critical gap in the literature. This emerging medium is now reaching a significant (and distinctly young) audience \cite{alexia}, and thus has great potential in terms of ideological messaging power. Two authors of the webcomics under my study were willing to provide traffic numbers --- Cyanide and Happiness claims an average of around 20 million pageviews a week, while Sunday Morning Breakfast Cereal claims about 1 million views per day (explosm.net ``general inquiries'', personal correspondence, April 2 2013, \& Zach Weiner, personal correspondence, April 2 2013). Compared to broadcast television which has viewership between five and ten million for the most popular programs \cite{hollywood}, webcomics clearly have numerically significant audiences.

The effect of popular media on public attitudes and perception is well researched, and indeed is a core focus of media studies as an academic field. James Lull \citeyear{lull2003hegemony} has consolidated much of this research using the overarching concept of ``hegemony'', a framework by which institutions of power within a society use media to cultivate specific ideologies. Drawing on the Marxist theory of the worker/owner dialectic, hegemony is a broader means of understanding how power structures are maintained. Fundamentally, Lull argues, hegemony is perpetuated by presenting and framing images about the world in such a way as to make structural power imbalances seem obvious, natural, and invisible in as many regards as possible. Thus, classes of people are structurally disempowered and may internalize their own disempowerment. Though this is not a complete and total control over the thoughts of individuals, hegemonic images are reinforced across a multitude of broadcast channels, creating an almost inescapable climate where minority status of individuals is constantly reaffirmed and normalized. Lull does note that gaps in hegemonic messaging exist, but the very presence of a subversive text is inherently marked as deviant. Counter-hegemonic texts \emph{can} catalyze and drive societal change, but only when such messages gain a sufficient level of popular backing. \cite{lull2003hegemony}.

This simultaneous strength and fragility of hegemonic dominance makes understanding the specific messages broadcast by media texts critical. If Lull is correct in that counter-hegemonic messages are critical for societal change, then the implicit messages in both mainstream and counter-cultural texts can provide descriptive and possibly even predictive information about broad social attitudes.

Larry Gross is another scholar who has created frameworks for understanding how hegemony affects public perception of minority groups. In ``Living with television: The violence profile'' \cite{gerbner1976living}, the concept of \emph{symbolic annihilation} is introduced. This is a process where certain minority groups are rarely if ever presented as subject in media, and thus are fundamentally removed from consideration as members of society. Such creation of invisibility results in annihilated groups having almost no social power.

In ``Out of the Mainstream: Sexual Minorities and the Mass Media'' \cite{gross} applies concepts of symbolic annihilation to gays and lesbians. Gross makes specific note of how gays and lesbians are both ``invisible'' minorities (e.g. not immediately identifiable based on external traits such as skin color) and present a perceived threat to the ``natural'' (i.e. heterosexual) societal framework. Additionally, Gross argues that broadcast television serves to define the social mainstream, and that regardless of program genre, broadcast television presents an increasingly ideologically unified message.

Regarding gay and lesbian people, this message is one of systematic repression and stereotyping. Gays and lesbians are given no narrative voice of their own, and are subject to continual villification or victimization when present at all. No texts portray gays and lesbians in positions of relative normalcy, and thus they are marked as a perpetual Other \cite{gross}.

It is important to note, however, that Gross was writing in 2001, and that portrayals of GSMRs on television may well have significantly broadened over the last decade. Dana Heller \citeyear{visibility} has specifically demarcated queer television studies into the pre- and -post visibility eras; the dividing line is April 1997 when Ellen DeGeneres came out as lesbian both personally and through her television character. Since then, numbers of queer characters on broadcast television have blossomed. Though Hellner conducts an extensive survey of queer television scholarship, the question of whether actual \emph{representations} of queer characters progressed or grown any more positive is left unanswered. Thus is seems reasonable to conclude that queer representations in mainstream media remain mixed at best.

Since webcomics as a media platform are by definition connected to the internet and internet culture, understanding the relationship between GSRMs and internet culture is essential for any meaningful analysis. Of course, the internet is anything but monolithic, but regarding internet-mediated spaces and dominant cultures therein, certain definite themes emerge. Roli Varma \citeyear{varma2007women} writes `` `Geek culture' evokes a high-tech, andocentric, sub-cultural milieu often associated with
computing.'' Varma describes geek culture as being fundamentally masculine, going as far as referring to abstract individuals with male pronouns despite the paper's focus on women in computing. When interviewed, women in computer science programs repeatedly note exclusion and belittlement by male peers and instructors. \cite{varma2007women}. 

In a study of 339 undergraduate students from the University of New Hampshire, Jerry Finn \citeyear{finn2004survey} found that while women and men experienced similar rates of harassment online, GLBT students were harassed with twice the average frequency. This suggests that in many online spaces, being a woman is accepted but queerness is not.

The webomic Penny Arcade is specifically focused on computer gaming culture. Taylor, Jensen, \& de Castell \citeyear{taylor2009cheerleaders} have conducted ethnographic research on how women position themselves in professional gaming spaces (specifically, competitive \emph{Halo 3} tournaments). The authors claim that opportunities for women to participate in gaming spaces are structurally limited, and thus women are not only rare in such spaces, but also tend to cluster around certain specific archetypes. They write ``competitive gaming [is positioned] as the exclusive domain of young men who best embody a `hypermasculine' subject position. The performance of pro-gamer masculinity is premised on technological mastery." \cite{taylor2009cheerleaders}. According to the authors, the only consistently socially-sanctioned role for women is competitive gaming spaces is that of ``booth babe'', conventionally-attractive and sexualized women hired as a form of corporate promotion. These ``booth babes'' are positioned around corporate demos and information tables, and serve as visual accompaniment to a marketing environment, but not agents directly interacting with event attendees. Additionally, Taylor et al. argue that the presence of ``booth babes'' at competitive gaming events serve to reaffirm the heterosexuality of participants against a perceived threat by a definitively male-homosocial environment \cite{taylor2009cheerleaders}.

\section{Research}

\subsection{Comics Studied}
In order to understand how gender, sexual, and romantic minorities are represented in webcomics, the five most popular webcomics were examined. In descending order of readership, these are xkcd, Penny Arcade, Cyanide and Happiness, Questionable Content, and Sunday Morning Breakfast Cereal \cite{alexia}. Though technically the comic Dilbert is ranked higher than Sunday Morning Breakfast Cereal, it has been excluded from review since it has been traditionally published via newspaper syndication rather than having been created as web-exclusive content. The review sample consisted of all strips published between March 1 2012 and March 1 2013, representing approximately 1250 individual strips reviewed. 

XKCD\footnote{Regarding proper capitalization of the comic's name, Munroe writes ``For those of us pedantic enough to want a rule, here it is: The preferred form is "xkcd", all lower-case. In formal contexts where a lowercase word shouldn't start a sentence, ``XKCD'' is an okay alternative. ``Xkcd'' is frowned upon.'' \cite{xkcd}}, authored by Randall Munroe, describes itself as ``A webcomic of romance, sarcasm, math, and language." \cite{xkcd}. However, I find this description incomplete; the subject matter of individual comics is quite broad, including science humor, infographics, social commentary, surrealist humor, and occasional pieces that challenge print comic conventions entirely (for example, in comic \#1110 ``Click And Drag", the last panel is actually a small section of a vast and explorable landscape). In general the art of Munroe's comics tends towards minimalism, with characters represented by occasionally-adorned stick figures and strips themselves drawn in a basic black-on-white inklines style. The comic's name is not an acronym, but rather four letters deliberately chosen to be unique and phonetically unpronounceable. XKCD updates every Monday, Wednesday, and Friday. 

With the first strip published in 1998, Penny Arcade (PA) represents one of the longest-running webcomics on the internet. The comic is co-authored by Jerry Holkins and Mike Krahulik, writing under the pseudonyms of ``Tycho'' and ``Gabe'' respectively \cite{pa}. These personas exist in the comic as stylized representations of the authors, and form the two core cast members. Penny Arcade rarely has any continuity between across individual strips, and generally is published in panel formats similar to traditional newspaper comics. PA is solidly rooted in gaming culture, and most strips have reference gaming either directly or by allusion --- it would be safe to claim that Penny Arcade's target audience is young male gamers. New comics are posted every Monday, Wednesday, and Friday. 

Cyanide and Happiness (C\&H) is a collaborative production between authors Kris Wilson, Rob DenBleyker, Matt Melvin, and Dave McElfatrick. C\&H features a minimalist art style, and is focused on surrealism, black comedy, and other humor generally regarded as obscene/offensive \cite{ch}. Though there is no regular update schedule, new comics are posted approximately daily. 

Questionable Content (QC) is a creation of Jeph Jaques, and is the only comic under review that features a definite canon and significant continuity between strips. The comic mostly focuses on social interactions between a large set of core characters over the course of several years. The website's official ``Cast'' page describes one character, Martin, as ``The closest thing to a protagonist we've got. Mopey, passive, unsure of what to do with his life. \textsc{sounds like a real winner, huh}'' \cite{qc}. While almost every strip contains some sort of joke or punchline, comics often focus on serious subject matter. Questionable Content updates every weekday, though occasionally non-canonical filler strips or guest comics to allow author vacations or accommodate for unexpected technical/medical problems. Though the setting is quite similar to modern-day Boston in most regards, it incorporates sci-fi elements such as human-companion AIs and a character who was born on space station.

Sunday Morning Breakfast Cereal (SMBC), created by Zach Weiner, is approximately equal parts observational humor, science, surrealism, interpersonal dynamics, human sexuality, and unexpected uses of logic. One often-used trope is to take a premise (either well-known or novel) and take that premise to an unusual or unexpected conclusion. For example, strip \#2584 describes how mastery of developmental biology results in humans demanding ever-increasing genital size to the point where an entire generation is nothing but walking genitalia \cite{smbc}.

In the interests of full disclosure, I should state that I consider myself a fan of xkcd, QC, and SMBC. These are comics I personally enjoy and read on a regular basis.\footnote{I suspect that this is due to these comics having significantly more positive portrayals of GSRMs, as demonstrated below. As a queer individual, this is important to my enjoyment of a work.} 

\subsection{Methodology}
The overall goal of this study is to evaluate representations of GSRMs in webcomics, particularly whether such representations are ones that might further societal acceptance of th groups in question. Since four out of the five comics studied have little to no continuity between strips, one approach taken to assess their representations of Gender, Sexual, and Romantic minorities is a quantitative count of their appearances in individual strips. This provides a quantitative measure of the penetration of GSRMs in this emerging medium. 

Additionally, it is important to note how particular individuals are portrayed. For example, while a comic might hypothetically have women present in almost every strip, they might exist purely as objects relative to men. 

The four non-continuity webcomics (xkcd, Penny Arcade, Cyanide and Happiness, and Sunday Morning Breakfast Cereal) were read sequentially in order of publication date, and the strips in which GSRMs appear was recorded. Recorded data included the comic number or date, what GSMRs were present, a one-line summary of the strip, and whether the strip seemed to warrant in-depth qualitative analysis. This provided a rough first-pass overview of the comics in question, and prompted more specific quantitative codings.

It was immediately (and not unexpectedly) obvious that appearances of women far outnumbered that of all other GSRMs. To further understand the role women filled in the webcomics studied, their representations were further coded into four primary categories: \emph{Professional}, \emph{Neutral Agent}, \emph{Object}, and \emph{Incidental Mention}.

The category \emph{Professional} is used to denote women who appear in positions of social power or are in vocations requiring a significant technical skillset. For example, women coded as \emph{Professional} would include  scientists, corporate executives, flight attendants, or politicians. \emph{Neutral Agents} are women that demonstrate subjectivity and agency in a particular strip, but either do not have a clearly defined position or are in a position of lower social regard than \emph{Professionals}. Women classified as \emph{Object} are distinct from agents in that they are not actors but rather acted upon. Also falling into this category are women that are reduced to objects of sexual desire. Finally, the category \emph{Incidental Mention} refers to women who have no real role in the strip itself and are either referred in passing or simply serve as background decoration.

 Regarding GSRMs other than women, my analysis is fundamentally informed by the assertions of both Gross \citeyear{gross} and Kessler\citeyear{queermediocrity} that positive representations of queer people are ones that normalize rather than tokenize queer individuals. Under such a framework, representations are considered positive if the particular minority status is incidental or not commented upon, and negative if it serves as punchline or is marked as deviant. Therefore, representations of queer individuals in webcomics have been coded into the categories of \emph{Neutral} and \emph{Joke/Othered}. For example, a strip consisting of a gay couple having an argument would be coded as \emph{Neutral} if their gayness is incidental to the comic, but \emph{Joke/Othered} if the humor is dependent on their gayness and/or they are marked as undesirably deviant. Gay/lesbian individuals and other GSMRs have been tabulated separately, since through the former two groups have gained widespread social acceptance, other groups are often far less recognized or understood. 

 Obviously, individuals may be coded multiple times within on strip. For example, if a strip features a polyamorous lesbian, appropriate coding would apply for her role as a women, her representation as a lesbian, and her portrayal as a polyamorous individual. 

Qualitatively, attention was payed to any overall trends in portrayals not revealed by simple quantitative analysis. Specific strips have also been analyzed on the bases of particularly progressive or problematic portrayals of Gender, Sexual and Romantic Minorities, commentary on GSMRs and their relationships with the subculture of a comic's target audience, and depictions of GSMRs that transcend traditional tropes. Focus was also given to strips that might contain highly multiple or contradictory readings. Qualitative analysis is inherently subjective, and thus it is difficult to provide any clear and consistent standard by which to determine what merits attention. 

Since Questionable Content features strong continuity and a core cast, quantitative counts of representations would have little to no analytical meaning, and thus it has been examined in a purely qualitative manner. 

\subsection{Results}

\begin{table}[!htbp]
\caption{Quantitative representations of GSMRs in non-continuity webcomics}
\label{quantitative}
    \begin{tabular}{l|llll}
    ~                               & xkcd     & PA       & C\&H      & SMBC      \\ \hline
    Number of strips reviewed       & 157      & 157      & 346      & 366       \\ \hline
    \# of strips with Women          & 48       & 27       & 115      & 236       \\
    Woman as Professional (\%)       & 14 (29\%) & 3 (11\%)  & 2 (2\%)   & 82 (34\%)  \\
     Woman as Neutral Agent (\%)     & 36 (75\%) & 12 (44\%) & 62 (53\%) & 148 (62\%) \\
    Woman as Object (\%)             & 1 (2\%)   & 9 (33\%)  & 30 (25\%) & 15  (6\%)  \\
    Woman as Incidental Mention (\%) & 0 (0\%)   & 3 (11\%)  & 17 (15\%) & 6 (3\%)    \\ \hline
    \# of strips with gays/lesbians  & 1        & 8        & 13       & 19        \\
    Neutral (\%)                     & 1 (100\%) & 1 (12\%)  & 2 (15\%)  & 16 (84\%)  \\
    Joke/Othered (\%)                & 0 (0\%)   & 7 (88\%)  & 11 (84\%) & 3 (16\%)   \\ \hline
    \# of strips with other GSMRs    & 2        & 4        & 7        & 6         \\
    Neutral (\%)                     & 2 (100\%) & 0 (0\%)   & 0 (0\%)   & 3 (50\%)   \\
    Joke/Othered (\%)                & 0 (0\%)   & 4 (100\%) & 7 (100\%) & 3 (50\%)   \\
    \end{tabular}
\end{table}

Table \ref{quantitative} shows quantitative counts for various groups using the coding scheme defined in the Methodology section. There are quite obviously distinct differences between different comics in terms of their representations of Gender, Sexual, and Romantic Minorities.

\subsubsection{Women}

The only comic in which a majority of strips contain women is Sunday Morning Breakfast Cereal, and almost all of their appearances are as either \emph{Professionals} or \emph{Neutral Agents}. For example, comic \#2672 \cite{smbc} features a variety of women, including a female quantum physicist and a female politician. Indeed, reading SMBC I have the general feeling that the ethnicity and gender of characters is essentially picked at random, since there seems to be no distinct pattern of women in positions significantly different than that of men. SMBC is also the only comic studied with a significant number of people of color (including women of color) present --- many of whom are in positions of political or social power. 

I suspect that this is a deliberate move by the author; by deliberately de-correlating gender, ethnicity, and social position, Weiner effectively portrays an egalitarian vision of the world. In the few cases where women exist as object, they often serve as a foil to mock men, especially in the domains of relationships or sexuality. One example is comic \#2683, discussing ``pickup lines that only sound good.'' In this strip, a man prepositions a woman by saying ``Baby, if sexual satisfaction is your disease, I'm you're cure tonight.'' \cite{smbc}. Though this clearly positions the woman as object --- the pick-up line is the only dialog or action in the strip --- the end effect is far more a reflection of the man than a statement about the woman. 

Another interesting trend in SMBC is the normalization of female sexuality and sexual desire. For example, in strip \#2617 a woman mourns her vibrator breaking in almost Shakespearean language, and in strip \#2706 a woman directly asks a man for clitoral stimulation, as setup for a joke about the ineffectiveness of ``alternative medicine'' \cite{smbc}.

Similarly, women in xkcd\footnote{Due to characters in xkcd being drawn as minimalist stick figures, hair length is the only visual cue regarding gender. It is generally regarded by both fans of the comic and casual observers that long hair signifies that a character is a woman.} are often represented as being technically competent. In strip \#1145, a mother (explicitly described as a scientist) is asked ``Why is the sky blue?" by her daughter, and gives a very precise technical answer. Again, there seems to be an overall effort made to portray roughly equal numbers of women and men as technically skilled --- a direct counter to the low number of women actually working in the sciences \cite{varma2007women}. 

Strip \#1111 in particular is a subversion of mainstream culture archetypes about women. In this strip, a female television reporter at a major Hollywood evades questions regarding the event itself and instead gives a grim outlook on the sun's eventual supernova and the inevitable heat-death of the universe \cite{xkcd}. Strip \#1027 directly addresses and ridicules ``pickup artists'', and is worth quoting at length:
\begin{quote}
Man 1: ``I've been learning tricks from pickup artists forums.''

Man 2: ``Pickup artists are dehumanizing creeps who see relationships as adversarial and women as sex toys.''

Man 1: ``No, it's just a bunch of tips! Like \emph{negging}: you belittle chicks to undermine their self-confidence so they'll seek your approval.''

Man 2: ``Just talk to them like a fucking human being.''

Man 1: ``Nah, that's a sucker's game. Ok --- wish me luck!''

<<Man 1 approaches a woman seated at another table>>

Man 1: ``You look like you're on a diet. That's great! How's the fruit plate?''

Woman: ``Ooh --- are we negging? Let me try! You look like you're going to spend your life having one epiphany after another, always thinking you've finally figured out what's holding you back, and how you can finally be productive and and creative and turn your life around. But nothing will ever change. The cycle of mediocrity isn't due to some obstacle. It's who you \emph{are}. The thing standing in the way of your dreams is that the person having them is \emph{you}. Ok, your turn! Ooh, try insulting my hair!''
\cite{xkcd}.
\end{quote}
The above exchange explicitly critiques and belittles a deeply sexist and misogynistic practice. More than that, the woman directly reverses the attempted creation of a power imbalance and gives an steadfast display of her own self-regard and agency.

 Women are scarce in Penny Arcade, and when present are only sometimes in positions of agency. They are represented as \emph{Object} in 33\% of the strips studies, the highest rate out of the four non-continuity comics. This is consistent with the work of Taylor et al. describing ``booth babes'' as sexualized eye candy for male gamers. This trend of reducing women to their sexual utility and placing them as an unreachable Other is most prominently displayed by the comic for October 15, 2012. This strip consists entirely of cover art for a fictional ``Official Strategy Guide to the Vagina.''\footnote{Some men do have vaginas and some women do not, but I highly doubt that such variability is a meaningful factor in this text.} Prominent is the head of a large, toothy, pink dragon. Boldface text loudly advertises ``Its Exact Location!, Detailed Maps of Every Zone!, Boss Encounters!, and All Hidden Areas!"\cite{pa}. Though it's likely that this comic is at least in part satirical, I still find it deeply problematic and emblematic of heterosexual male gaze in a culture that views relationships as adversarial. 

The strip for January 1, 2013 is difficult to analyze due to multiple discursive threads being deployed simultaneously. To quote:
\begin{quote}
Tycho: I think Cheeto's girlfriend is one of those \emph{fake geek girls}.

Gabe: I'm going to use some words you might not be familiar with, but I want you to promise me that you'll look them up when I'm done. Listen up, Satan. Who are you to assess her credentials as a woman \emph{or} a gamer? There's no gate, you're not the gatekeeper, so go drink some bad poison and \emph{shit out your life.}
\end{quote}

However, in the third and final panel, it is revealed that the girlfriend in question is literally fake --- a scarecrow constructed from assembled twigs, dressed in a batman shirt and adorned with lipstick \cite{pa}. Initially, Gabe's rant reads like a literal and strong rebuke of geek culture's tenancy to scrutinize every aspect of women in geek spaces for ``legitimacy.'' However, the reveal that the woman in question is fake may simply be serving to make Gabe appear the fool and mock vocal advocates for inclusion. Reading this isolated strip without deeper narrative context results in having no substantial footing to definitely support either claim.

In Cyanide and Happiness, women are only presented as professionals in two instances. The rate of women as \emph{Object} is slightly lower than that in Penny Arcade, but the individual representations are potentially more problematic. A typical example is strip \#2779, in which a father answers a son's question ``Is the earth round of flat?" by comparing the round breasts of his new girlfriend as distinct improvement over the flat breasts of the kid's mother. The father in particular describes his past-self as ``stupid'' for thinking that flat was right \cite{ch}. It is plain that this strip reduces the value of a women to a single sexualized attribute. In strip \#3018, a man lets a homeless woman live in his house despite a friend's objections (which are, incidentally, incredibly stereotypically negative about the homeless). The man allows the woman to stay on the grounds that ``She miiiiiiiight suck my dick." In the next panel, a young student is shown concluding ``and that's why there aren't very many homeless women.'' \cite{ch}. I find this strip deeply problematic, not only because of the reduction of a woman to a sex object, but also because homeless women are at tremendous risk of victimization and sexual assault \cite{wenzel2000antecedents}.

Women are often presented as \emph{Neutral Agents} in C\&H, but the majority of their representations are defined in terms of a relationship with a man. The most common of these are women cast in the role of girlfriend. Men in C\&H are regularly shown as independent agents, but it is relatively rare for a woman to be portrayed as narratively unrelated to a man. This effectively reduces women's demonstrated agency and independence. 

However, a few comics in Cyanide and Happiness offer counter-hegemonic messages regarding gender. In strip \#3069 a woman corrects two men on their misconceptions of feminists as bra-burning man-hating women who don't shave; when she's jokingly implied to be bisexual by the two men, she replies ``We definitely hate \emph{some} men, though\ldots" C\&H strip \#2861 shows a man offering platter of ``traditional gender rolls'', said to be baked by his wife that morning. When another man takes one, he comments ``These are pretty awful'', to which the first man replies ``Yeah, she should leave the cooking to me.'' \cite{ch}. This is an effective use of pun and unexpected humor to mock traditional heterosexist culture. 

\subsubsection{Other GSRMs}
Overall, representations of Gender, Sexual, and Romantic minorities other than women were relatively scant. However, for most of the comics reviewed, there was enough representation for trends to appear. 

In Penny Arcade, 88\% (seven out of eight) strips which include gay and lesbian characters used their status as GSRMs for punchlines. For example, the comic posted for September 12, 2012, features Gabe mocking stereotypically effeminate gay behavior. The comic for May 28, 2012 has ``gay muslims plant a dirty bomb inside the internet'' as an example of bad randomly-generated videogame threat, seeming to imply that being gay and Muslim is inherently contradictory \cite{pa}. GSMRs other than gays and lesbians are alluded to rather than being directly represented, but such allusions are consistently degenerating. October 31, 2012's comic, for example, uses ``Don't forget to get tested for AIDS'' as an example of obvious advice for the sake of a punchline, trivializing the very real struggles of HIV-positive individuals \cite{pa}.

Cyanide and Happiness likewise displays generally negative attitudes towards queer characters. In strip \#2756, when a son asks to be taught how to fight in order to stand up to bullies, the father advises him to act in a stereotypically effeminate manner and directly proclaim gayness, receive a beating, then win a lawsuit over being the victim of a hate crime. Strip \#2801 uses an unwanted gay advance as humor, and in strip \#2783 a man having regular sex with another man is used as a punchline \cite{ch}. These examples are all fairly typical for the comic. Some other GSRMs represented include a crossdresser, a boy sexually attracted to trees, HIV-positive people, non-normative sex and relationships, a man romantically attached to a blow up doll, and a trans* man. Each of these are used as punchline.The trans* character mentioned appears in strip \#2914, claiming ``I feel like a man trapped in a woman's body --- I want to have my breasts removed.'' The doctor replies with a diagnosis of malignant breast cancer \cite{ch}. This strip is drawn in a way that specifically gives the trans* man a feminine-cued appearance, perhaps intending to have the character coded as a woman by the audience. 

With only three representations of visibly queer characters in xkcd, attempting to draw conclusions about overall trends in the comic would be nonproductive.

Sunday Morning Breakfast Cereal is thus alone in having gays and lesbians given predominantly neutral representations, perfectly embodying ``just plain gay people'' as Gross \citeyear{gross} advocated. For example, strip \#2639 features a father playing mind-games with his young son to reduce their household electricity usage, with the last panel showing the two male parents having coffee at a dining room table. Similarly, strip \#2541 features a lesbian couple in which one woman complains about lack of communication regarding sex \cite{smbc}. In both of these examples, the queer relationship structure is completely incidental to the humor, thus serving to mark the queerness as normal rather than deviant and deserving of commentary. In other cases, cultural attitudes towards gays and lesbians are directly subverted, as in strip \#2781. In this comic, the president mobilizes ``an elite force of America's gayest individuals,'' ``a lightning rod for God's wrath'' as a special unit using their powers for good, resulting in changing the course of hurricanes and generating ``sodomy induced lightning strikes" as a source of cheap and nonpolluting electricity \cite{smbc}.

\subsubsection{Questionable Content}
As Questionable Content features a clear canon, continuity, and core cast of characters, categorized numerical analysis of representations would not be illuminating. Thus, analysis here is qualitative and contextual, and examines specific situations and portrayals that arise in QC. 

The characters in Questionable Content represent an approximately even mix of men and women, all with distinctly realized agency and autonomy. Faye (one of the central characters) is seen welding together a dinosaur-shaped espresso maker she designed in strip \#2173, and in \#2345 is casually playing with a broadsword in the coffeshop where she's employed. In strip \#2345, Marigold --- a nerdy, slightly isolated young woman --- soundly trounces a personal rival in the computer game World of Warcraft. Further examples include Thai, a lesbian woman in charge of running a local library, and Dora, who owns a financially successful coffeeshop \cite{qc}. 

Within the review period, there are four plot arcs in QC that most clearly convey attitudes regarding Gender, Sexual, and Romantic Minorities. In chronological order these are: Hannelore and her friends attending her dad's birthday party aboard his space station, Thai pursuing a romantic relationship with Dora, Claire coming out as trans*, and the wedding of Martin's dad. The ensemble cast results in multiple sub-arcs being told simultaneously, and there is some amount of overlap between arcs. However, this does not prevent these plot threads from being critically analyzed. 

The first of these arcs is focused around Hannelore, a neurotic and obsessive-compulsive young woman, born on a space station to ultra-wealthy parents. She is invited to her father's birthday party aboard the space station, and invites her friends Martin and Marigold. In a spacious ballroom ``The greatest collection of brilliant minds in human history'' is gathered, and appears to be a roughly equal mix of men and women \cite{qc}. Overwhelmed by the technical conversations and large social space, Marigold retreats into a corner to play Pokemon on a handheld device. When a young male scientist expresses romantic interest in her, she is initially flustered and confused. However, by the next morning the two have engaged in copious makeouts, and in strip \#2164 she confesses to having ``Might've toched his weiner,'' implied to be her first sexual encounter. While her friends have definitely encouraged this development and are almost voyeuristically interested in her escapades, there seems to be no stigmatization of her lack of sexual experience. 

Meanwhile, the space station AI (named ``Station'' and projected as a holographic avatar) invites Hannelore to a private dinner date. Station then politely asks Hannelore to move back to living on the space station, expressing deep attachment and love for her, describing the slow process of how he watched her development, slowly untangling her neurosis and growing to know her as a person, then finally describing her as ``a friend, a sister, a daughter.'' Hannelore replies that apart from the companionship of Station, her life prior to moving to earth was ``full of dark and scary things,'' and describes how she now has a full and rich life on earth. Though Hannelore plans to return to earth, she expresses a similar deep love towards Station. \cite{qc}. Hannelore has already been decidedly established as asexual (in part due to her neurosis, but mostly out of the simple lack of desire), and her conversation with Station is a demonstration that asexual folks can and do live rich and fulfilling lives and do form deep interpersonal bonds. In fact, the very inclusion of Hannelore as a core character serves to normalize asexuality as part of natural human diversity. 

An ambiguous amount of time later, Thai decides to pursue a romantic interest in Dora. Dora (who is bisexual) had ended a relationship up with Marten some time ago. In strip \#2219, when Thai comments that ``making a move on Dora\ldots would be weird for Martin'' he replies ``Dora and I are cool now. If you wanna make out with her and she wants to make out with you, I'm ok with it.'' \cite{qc}. Not only does this establish consideration and consent as key parts of a relationship, it reveals that Martin is unpossesive towards a past girlfriend --- a positive representation of respect for women's individual agency. That same night, after a casual gathering at Martin's place, Thai asks Dora to walk her home, and in strip \#2227, Thai very directly expresses interest in Dora before heading off to bed.

The next morning, after a fair amount of hesitation and anxiety on the part of both women, Thai comes to meet Dora at the coffeeshop, pulls her aside, and in strip \#2246 says
\begin{quote}
``Okay, so about last night: I was drunk, but I totally meant it. I'm super into you. And not just physically, I mean the whole shebang. Massive crush over here. I'm about to barf butterflies. And if you're not interested, or if you really think you can't do a relationship, that's fine. I totally understand. But I need an answer. No second-guesses. No waiting around while you subconsciously convince yourself it wouldn't work out. `Cause if it doesn't work out, hey, that happens all the time. I wanna give it a try and see. So, uh  ---''
\end{quote}
Dora's response is to kiss Thai furiously and passionately, leaving them both wordlessly stammering and blushing \cite{qc}. Once again, this is an example of extremely positive relationship dynamics. Interest is expressed directly and unambiguously, and the other party is given full freedom to either try a relationship or not, rather than being pressured into anything. Additionally, I appreciate the understanding that some relationships don't work out and that's a natural part of things, not a reason to avoid intimacy altogether. 

Later, in strips \#2256 through \#2264, Dora and Thai spend an evening date together, including casual chatter, copious amounts of flirting, and a conversation about how they each came out as non-heterosexual to their parents \cite{qc}. The two women going out on a date is presented neutrally rather than fetishistically, and overall their developing relationship is strongly marked as being normal rather than deviant. This pattern continues throughout the rest of the sample set; while there are anxieties and insecurities about the relationship's development, they are characterized as typical of any romantic relationship, lesbian or not. 

In the third prominent arc, Emily (an intern at the library where Marten and Thai work) invites basically the entire core cast to a party at her parent's lake house. In strip \#2323, midway through the evening at the lake house, Claire (another intern) nervously tells Martin ``I\ldots I'm trans. And\ldots since we're friends, I'd thought you'd like to know that about me.'' Martin responds casually, thanking Claire for telling him but overall treating her being trans as not a huge deal. In the author's note for the comic, Jeph Jaques wrote 
\begin{quote}
``I have to admit, I am nervous about posting this comic, because including a trans person in my cast is something I have wanted to do for years and I really, really want to do a good job of it. One of the major themes of QC, I think, is of inclusion, and this seemed like a pretty important thing to include. I have given it a lot of thought and done a lot of research, so hopefully I won't screw up. I'll do my best, anyway.'' \cite{qc}
\end{quote}
This strikes me as an honest and heartfelt statement, and I especially appreciate how Jaques acknowledges that portraying a trans woman respectfully can be a difficult issue given our dominant cultural climate of transphobia. 

In the next strip, Claire volunteers some information about her transition. Marten then asks how open he should be about Claire being trans, to which she responds ``Uh, quiet but honest I guess? Like I'd prefer if you didn't just \emph{tell} everybody, but if they straight-up ask that's ok.'' \cite{qc}. As a trans woman myself, I find this portrayal both genuine and highly respectful. Marten's response to Claire coming out is exactly what I'd hope for upon telling someone that I'm trans*. Later, in strip \#2389 (during the wedding plot arc discussed below), Claire nervously asks Marten if she looks good in a particular blue dress, anxiety plain of her face. Again this strikes me as a genuine representation, since strongly gendered clothing such as dresses can be a significant source of trepidation, nervousness, and self-consciousness.

The fourth significant arc is the Martin's dad getting re-married, this time to a guy. Martin invites Claire to the wedding, in part because she's one of the few cast members who owns a car and he needs a ride. On the way to the wedding (strip \#2378), Marten briefs Claire on his family, saying ``Dad runs a fancy nightclub down in Miami. Dad's fiance Maurice is a golf instructor. Mom used to be a really well-known fetish model, and now she's a professional dominatrix.'' The night before the wedding, Marten, Claire, Henry (Marten's dad), Maurice, and Veronica (Marten's mom) enjoy a casual, amiable dinner together. The morning of the wedding, Veronica's longtime friend Jane arrives, and the two swap friendly jibes. Jane describes herself as an erotic photographer, then asks if she can take a photo of Veronica under the table since she could ``really use more material for [her] upskirt site.'' In the next strip (\#2393), Veronica and Jane leave to go take more photos, and Claire asks if Veronica's vocation bothers Marten. In the exchange that follows, Marten describes how his mom was open about it once he was old enough to understand, and he's simply used to it as part of his life \cite{qc}. This is yet another example of QC's pattern of normalizing what general society regards as deviant. 

Thus, overall, Questionable Content is in fact a distinctly positive portrayal of Gender, Sexual, and Romantic Minorities. Rather than being filled with otherizing portrayals, characters are shown to be fully realized humans each with their own distinct personalities and lives. 
\section{Conclusion}
It is clear that as an emerging medium, webcomics have significant variation in terms of their portrayals of Gender, Sexual, and Romantic Minorities. While Penny Arcade and Cyanide and Happiness often reduce GSRMs to objects or stereotypes, xkcd, Sunday Morning Breakfast Cereal, and Questionable Content deliver women in positions of agency and technical competence, as well as normalizing portrayals of queer characters and people who engage in non-normative sexual or romantic practices. SMBC and QC in particular fully accomplish the goal of presenting ``just plain gay folks'' as individuals. 

Webcomics reach numerically significant audiences, on par with those of television. In terms of discursive space, there is far more variability in webcomic ideologies than that of broadcast television. This potentially places the internet as a location where both countercultural and hegemonic messages are being simultaneously and conflictingly broadcast, often even within an individual work. In many ways this presents great opportunity for the internet as a platform for social change. Alternately, the wide variety of available messages and viewpoints may result in individuals only following internet sources that match their preexisting ideology, creating an atmosphere of polarization and entrenchment. Longterm effects of the increasing omnipresence of the internet remain to be seen, but I for one am optimistic about the internet as a progressive force in society.

Further avenues of research include more in-depth analysis regarding portrayals of women in webcomics, and the possible implications thereof. Another approach might be to examine popular webcomics through lenses of race, class, institutional power, or any of the other facets of kyriarchy. Additionally, there exist webcomics with decidedly queer or outside-of-maistream themes and casts, providing a fascinating opportunity to research emerging countercultural texts. Webcomics as a medium are ripe for analysis, and I sincerely hope that their status as an academically ignored topic will soon end. 

\subsection{Aknowledgements}
Writing this paper would not have been possible without deep and caring emotional and logistical support by the people around me. I would like to first thank my girlfriends; Emilia Azure, Erin Cowden, and Amy Wilhelm for their constant encouragement and belief in my abilities. Emilia deserves special mention for always supporting me when I needed it, as well as being the best editor and proofreader I know. Other people who have provided invaluable encouragement include Erika Frankel, Eli Dupree, and Creighton Grotbeck --- thank you all for reminding to take care of myself and to strike a balance between research and my own mental health. Finally, I would like to thank Myra Washington for her aid in academic research, focusing of my thoughts, never-ending compassion, and her drive to push me into researching and writing to the very best of my abilities.

\bibliographystyle{apacite}
\bibliography{gsrm-cite}


\end{document}